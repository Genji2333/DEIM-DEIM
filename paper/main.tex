\documentclass[lettersize,journal]{IEEEtran}
\usepackage{amsmath,amsfonts}
\usepackage{amssymb}
\usepackage{algorithmic}
\usepackage{algorithm}
\usepackage{array}
\usepackage[caption=false,font=normalsize,labelfont=sf,textfont=sf]{subfig}
\usepackage{textcomp}
\usepackage{stfloats}
\usepackage{url}
\usepackage{verbatim}
\usepackage{graphicx}
\usepackage{cite}
\hyphenation{op-tical net-works semi-conduc-tor IEEE-Xplore}
% updated with editorial comments 8/9/2021

% % %
\usepackage{fontspec}   % XeLaTeX 字体设置
\usepackage{xeCJK}      % 中文支持
\setCJKmainfont{SimSun} % 中文主字体(宋体)
\setCJKsansfont{SimHei} % 中文无衬线字体(黑体)
\setCJKmonofont{FangSong} % 中文等宽字体(仿宋)
\usepackage{orcidlink}

\usepackage{makecell}
\usepackage{booktabs}
\usepackage{multirow}
\usepackage[capitalize]{cleveref}
\crefname{section}{Sec.}{Secs.}
\Crefname{section}{Section}{Sections}
\Crefname{table}{Table}{Tables}
\crefname{table}{Tab.}{Tabs.}
\crefname{figure}{Fig gogogo.}{Figs.}

\usepackage{xcolor}
\definecolor{BestRed}{rgb}{0.8,0,0}
\newcommand{\best}[1]{\textcolor{BestRed}{#1}}
% % %

\begin{document}

\title{面向视觉任务的自适应小波卷积:结合坐标门控、条带频率门控与像素级融合}

\author{IEEE Publication Technology,~\IEEEmembership{Staff,~IEEE,}
        % <-this % stops a space
\thanks{This paper was produced by the IEEE Publication Technology Group. They are in Piscataway, NJ.}% <-this % stops a space
\thanks{Manuscript received April 19, 2021; revised August 16, 2021.}}

% The paper headers
\markboth{Journal of \LaTeX\ Class Files,~Vol.~14, No.~8, August~2021}%
{Shell \MakeLowercase{\textit{et al.}}: A Sample Article Using IEEEtran.cls for IEEE Journals}

% \IEEEpubid{0000--0000/00\$00.00~\copyright~2021 IEEE}
% Remember, if you use this you must call \IEEEpubidadjcol in the second
% column for its text to clear the IEEEpubid mark.

\maketitle

\begin{abstract}
卷积神经网络在目标检测、语义分割与图像复原等任务中广泛应用,但仅依赖固定尺度的局部卷积难以同时兼顾细节纹理与全局结构。小波变换提供了天然的多尺度、可解释的频带分解框架,能够将特征显式拆分为低频结构与高频细节,从而为频域建模提供便利。然而,现有小波卷积通常在子带上采用相对静态的逐通道卷积或简单重标定,缺乏对空间位置、方向性频率与跨分支信息的自适应控制,导致不同样本与不同区域的频带贡献难以动态调整。

本文提出一种接口与现有小波卷积兼容的自适应小波卷积模块(AWTConv2d)。该模块在保持小波分析/合成骨架不变的前提下,引入三类关键机制:其一,面向每个通道的子带可学习混合,通过按通道分组的 $1\times1$ 变换实现 $\{LL,LH,HL,HH\}$ 之间的可学习重组;其二,融入坐标门控与条带频率门控,使子带特征能够根据空间位置与方向性低/高频成分进行动态调制;其三,采用像素级自适应融合替代简单相加,将空间卷积分支与小波重建分支在像素层面进行软选择,提升跨域信息互补能力。

在\textbf{[数据集占位:例如 COCO / DOTA / VisDrone / 自定义数据集]}上的实验表明,AWTConv2d 在几乎不改变上层网络结构的情况下,能够稳定提升\textbf{[指标占位:mAP / AP50 / mIoU / PSNR 等]},并在消融实验中验证了各组件对性能的贡献。
\end{abstract}


\begin{IEEEkeywords}
小波变换,小波卷积,频域建模,坐标门控,注意力机制,特征融合
\end{IEEEkeywords}

%%%%%%%%% BODY TEXT
\section{Introduction}
\label{sec:intro}

Convolutional neural networks (CNNs) have achieved remarkable progress in object detection and dense prediction. For instance, the DETR family advances end-to-end detection through global matching paradigms~\cite{carion2020detr,zhu2020deformable}, and real-time detectors continue to be improved in both architecture and training strategies~\cite{zhao2024rtdetr}. Nevertheless, classical convolution operators are still centered around local receptive fields; their capability to model multi-scale patterns, directional textures, and frequency-band information is often obtained indirectly via deep stacking. In scenes with rich textures, large scale variations, or heavy background clutter, relying only on spatial-domain local modeling with fixed kernels may lead to detail loss or accumulated structural errors.

Frequency-domain approaches offer an alternative perspective. Compared with spatial-domain convolutions, frequency representations can more explicitly separate low-frequency structures from high-frequency details, providing better controllability for feature extraction and feature fusion. Recent studies have explored dynamic frequency filtering and frequency-aware fusion. For example, FFT-based dynamic filtering learns frequency-domain weights to adaptively mix responses across samples~\cite{dynamicfilter2023}, while frequency-aware feature fusion explicitly generates low-pass and high-pass kernels to enhance cross-scale fusion~\cite{freqfusion2024}. However, directly adopting FFT in practical detection/segmentation pipelines often faces challenges such as resolution-dependent frequency weights, complicated boundary handling, and non-trivial implementation overhead.

Wavelet transforms combine frequency interpretability with spatial locality. They decompose features into multi-scale pyramids of low-frequency subbands (structures) and high-frequency subbands (textures), and can map processed subbands back to the spatial domain via invertible reconstruction. Wavelet-based convolution modules (e.g., WTConv) provide explicit band pathways while remaining structurally simple~\cite{wtconv2024}. Yet, in many existing implementations, subband processing is relatively static, typically consisting of depthwise operations and fixed re-scaling. Consequently, the importance of different bands is hard to adjust dynamically across locations; directional low/high-frequency responses are not explicitly modeled; and the fusion between the spatial branch and the wavelet branch is commonly a simple summation, which may cause branch interference in early training.

To address these issues, we propose an adaptive wavelet convolution module, AWTConv2d, which enhances both wavelet-subband processing and cross-branch fusion. First, we introduce per-channel learnable subband mixing so that information exchange among $\{LL, LH, HL, HH\}$ is no longer restricted by fixed subband semantics. Second, inspired by coordinate gating~\cite{coordgate2024}, we generate spatial gates from normalized coordinates to modulate wavelet-domain features, explicitly modeling location-dependent band contributions. Meanwhile, we introduce strip frequency gating~\cite{fsa2024} to reorganize directional low/high-frequency components and strengthen the modeling of elongated textures and edge structures. Finally, we employ pixel-wise adaptive fusion~\cite{cga2024} between the spatial convolution branch and the wavelet reconstruction branch, avoiding conflicts caused by naive summation and enabling per-pixel selection of more reliable information.

Our contributions are summarized as follows. We present a plug-and-play adaptive wavelet convolution module that remains interface-compatible with existing wavelet convolutions. Without altering the rest of the backbone, the proposed coordinate gating, strip frequency gating, and pixel-wise fusion significantly improve adaptivity across samples and spatial regions. Extensive experiments on \textbf{[Dataset Placeholder]} with both main results and ablations validate the effectiveness of the proposed design.

\section{Related Work}
\label{sec:releated-work}

\subsection{Wavelet Transforms and Wavelet Convolutions}
Wavelet transforms decompose signals into subband representations across multiple scales and orientations through a set of analysis filters, combining frequency interpretability with spatial locality. Unlike FFT, wavelet decompositions preserve local support in the spatial domain, making them more suitable as modular operators embedded in convolutional networks. In vision tasks, wavelets have long been used for multi-scale representation and detail enhancement. Recently, wavelet analysis/synthesis has also been integrated into convolutional structures. For example, WTConv performs multi-level wavelet decomposition of features, applies lightweight convolutions on subbands at each level, and reconstructs features via inverse transforms, offering an explicit frequency-band pathway with low modification cost~\cite{wtconv2024}. Nevertheless, many existing wavelet convolutions remain relatively static in both subband processing and cross-branch fusion, limiting their adaptivity to complex scenes.

\subsection{Frequency-domain Modeling and Dynamic Filtering}
Frequency-domain modeling commonly enhances structure and texture representation by explicitly reweighting or filtering frequency components. Dynamic filtering learns content-adaptive frequency weights, enabling input-dependent filtering responses~\cite{dynamicfilter2023}. Frequency-aware fusion further focuses on low-/high-frequency complementarity across scales, generating low-pass and high-pass kernels to resample features and improve residual enhancement~\cite{freqfusion2024}. These works highlight the importance of frequency adaptivity, but they also reveal practical issues when directly adopting FFT or complex resampling operators in detection and dense prediction pipelines, such as resolution-dependent parameterization and increased engineering complexity.

\subsection{Spatial Gating and Position-conditioned Modulation}
Position-conditioned modulation maps external conditions (e.g., coordinates or semantic cues) to channel-wise or pixel-wise scaling factors, thereby enabling spatially varying responses. CoordGate proposes to generate gating weights for spatially varying convolutions from coordinate encodings, improving spatial adaptivity while keeping the computation efficient~\cite{coordgate2024}. Related conditional modulation approaches include FiLM~\cite{perez2018film} and AdaIN~\cite{huang2017adain}, which more generally demonstrate that learnable feature modulation can effectively enrich model expressiveness.

\subsection{Attention Mechanisms and Feature Fusion}
Attention mechanisms explicitly model channel, spatial, or pixel-wise weights to emphasize informative content. For fusing two-branch or multi-branch features, pixel-wise attention provides a finer-grained selection capability. CGA-style fusion computes channel attention and spatial attention to guide pixel attention, achieving per-pixel soft fusion between two feature streams~\cite{cga2024}. This idea is aligned with our motivation: when the spatial convolution branch and the wavelet reconstruction branch exhibit different reliability across regions, pixel-wise fusion can mitigate the interference caused by naive summation.

In summary, our method introduces dynamic frequency modulation and spatial conditioning into the wavelet analysis/synthesis scaffold, and adopts pixel-wise fusion to strengthen cross-branch complementarity, improving adaptivity while preserving plug-and-play usability.

% \IEEEpubidadjcol
\section{Method}
\label{sec:method}

This section first reviews the basic form of wavelet convolution and then details the proposed adaptive wavelet convolution module, AWTConv2d. The module is designed to remain interface-compatible with WTConv2d, so it can be used as a drop-in replacement for depthwise separable convolutions or depthwise convolutions in existing detection/segmentation networks.

\subsection{Preliminaries: 2D Discrete Wavelet Analysis and Synthesis}
Given an input feature map $\mathbf{X}\in\mathbb{R}^{B\times C\times H\times W}$, the 2D discrete wavelet transform can be viewed as a downsampling convolution with a set of fixed analysis filters. The output consists of one low-frequency subband and three high-frequency subbands. We denote wavelet analysis as
\begin{equation}
\mathbf{U}=\mathcal{W}(\mathbf{X})\in\mathbb{R}^{B\times C\times 4\times \frac{H}{2}\times \frac{W}{2}},
\end{equation}
where $\mathbf{U}_{:,:,0,:,:}$ corresponds to the $LL$ subband and $\mathbf{U}_{:,:,1:4,:,:}$ corresponds to the three high-frequency subbands $LH/HL/HH$. Wavelet synthesis (inverse transform) is denoted as
\begin{equation}
\hat{\mathbf{X}}=\mathcal{W}^{-1}(\mathbf{U})\in\mathbb{R}^{B\times C\times H\times W}.
\end{equation}
In implementation, $\mathcal{W}$ and $\mathcal{W}^{-1}$ can be realized by grouped convolution and transposed convolution, respectively. The filters are determined by the chosen wavelet basis (e.g., db1/db2) and are kept fixed.

\subsection{Baseline: Subband Processing and Fusion in WTConv2d}
The basic pipeline of WTConv2d performs multi-level wavelet decomposition, applies depthwise convolution on subband features, reconstructs features through inverse transforms level by level, and finally adds the reconstruction to a spatial depthwise convolution branch~\cite{wtconv2024}. Its key advantage is an explicit band pathway with invertible reconstruction. However, the subband processing is usually static (depthwise convolution plus fixed scaling), and the fusion is a simple summation, lacking input-adaptive selection mechanisms.

\subsection{AWTConv2d: Adaptive Subband Token Mixer}
AWTConv2d keeps the wavelet pyramid scaffold but upgrades the per-level subband processing into a composite operator consisting of local refinement, learnable subband mixing, spatial/directional modulation, and subband attention. For the wavelet output at level $l$, denoted as $\mathbf{U}^{(l)}$, we first reshape it into
\begin{equation}
\mathbf{T}^{(l)}\in\mathbb{R}^{B\times 4C\times H_l\times W_l},
\end{equation}
where $H_l=H/2^l$ and $W_l=W/2^l$.

\paragraph{(1) Local refinement within each subband}
We apply depthwise convolution to $\mathbf{T}^{(l)}$ to capture local patterns inside each subband, denoted as $\phi_{dw}(\cdot)$.

\paragraph{(2) Per-channel learnable subband mixing}
To enable information exchange among $\{LL,LH,HL,HH\}$ within the same channel, we introduce a grouped $1\times1$ transform $\phi_{mix}(\cdot)$ with groups set to $C$. This learns a $4\rightarrow 4$ linear mixing inside each channel. Compared with mixing across all $4C$ channels, this design avoids unconstrained cross-channel coupling, improving stability and interpretability.

\paragraph{(3) Coordinate gating: position-conditioned band modulation}
The dependence on low-/high-frequency information typically varies across spatial locations. For example, boundary regions may rely more on high-frequency details, while large smooth background regions may rely more on low-frequency structures. Inspired by CoordGate~\cite{coordgate2024}, we use the normalized coordinate grid $(x,y)\in[-1,1]^2$ as conditioning input, and employ a lightweight $1\times1$ convolutional network to produce a pixel-wise gate
$\mathbf{G}_{coord}^{(l)}\in(0,1)^{B\times 4C\times H_l\times W_l}$. Then we modulate the features by
\begin{equation}
	ilde{\mathbf{T}}^{(l)} = \mathbf{T}^{(l)} \odot \mathbf{G}_{coord}^{(l)}.
\end{equation}
This implementation does not depend on a fixed input resolution and thus can be applied across stages and varying input sizes.

\paragraph{(4) Strip frequency gating: directional low/high-frequency re-organization}
To explicitly model directional textures and elongated structures, we introduce strip-based low/high decomposition~\cite{fsa2024}. Concretely, we perform strip average pooling along the horizontal and vertical directions to obtain low-frequency components, and compute high-frequency components as residuals. Learnable coefficients then mix low/high components and are injected back in a residual manner. Denoting this operator as $\psi_{strip}(\cdot)$, we have
\begin{equation}
\bar{\mathbf{T}}^{(l)} = \psi_{strip}(\tilde{\mathbf{T}}^{(l)}).
\end{equation}
Unlike relying solely on convolution kernels to implicitly learn directional responses, this mechanism provides structured control over low-frequency smoothing and high-frequency edge emphasis.

\paragraph{(5) Subband attention: content-adaptive reweighting across subbands}
After subband mixing and spatial/directional modulation, we apply subband attention to $\bar{\mathbf{T}}^{(l)}$. The attention extracts global statistics via global average pooling, and uses a grouped $1\times1$ transform with groups=$C$ to generate four subband weights per channel, enabling content-adaptive selection among $LL/LH/HL/HH$.

Overall, the level-$l$ subband processing can be summarized as
\begin{equation}
\mathbf{T}^{(l)} \leftarrow \alpha\big(\psi_{strip}(\phi_{coord}(\phi_{mix}(\phi_{dw}(\mathbf{T}^{(l)}))))\big),
\end{equation}
where $\alpha(\cdot)$ denotes the combination of subband attention and scaling.

\subsection{Wavelet Reconstruction and Pixel-wise Cross-branch Fusion}
After processing $L$ levels of subbands, we perform top-down inverse transforms to reconstruct the wavelet branch output $\mathbf{X}_{wave}$. In parallel, the spatial branch produces $\mathbf{X}_{base}$ using depthwise convolution.

Prior approaches often fuse branches by a direct sum $\mathbf{X}_{base}+\mathbf{X}_{wave}$, but the reliability of the two branches can vary across regions. Therefore, we introduce a pixel-wise fusion module. Inspired by content-guided attention fusion~\cite{cga2024}, we jointly use channel attention and spatial attention to generate a pixel gate $\mathbf{P}\in(0,1)^{B\times C\times H\times W}$, and perform per-pixel soft fusion:
\begin{equation}
\mathbf{X}_{fuse} = \mathbf{X}_{init} + \mathbf{P}\odot\mathbf{X}_{base} + (1-\mathbf{P})\odot\mathbf{X}_{wave},\quad \mathbf{X}_{init}=\mathbf{X}_{base}+\mathbf{X}_{wave}.
\end{equation}
In addition, we keep a conservative channel-wise gate $\sigma(\mathbf{g})$ to modulate the magnitude of $\mathbf{X}_{wave}$ before fusion, improving training stability when the wavelet branch is initially under-optimized. The final output uses a $1\times1$ projection to match the channel dimension required by the downstream network.

\subsection{Complexity and Plug-and-play Discussion}
Compared with WTConv2d, the added overhead of AWTConv2d mainly comes from the grouped $1\times1$ subband mixing, the lightweight coordinate gating network, strip pooling operations, and the depthwise $7\times7$ convolution used in pixel-wise fusion. All components can be efficiently implemented with standard deep learning operators without custom CUDA kernels. Meanwhile, AWTConv2d preserves the same input/output interface as WTConv2d, allowing straightforward replacement and ablation in existing codebases.

\section{实验}
\label{sec:exp}

本节介绍实验设置、对比方法与结果分析。由于本文工作以模块替换为主,我们保持主干网络、训练策略与数据处理流程尽量一致,仅替换指定位置的卷积模块以验证 AWTConv2d 的通用增益。

\subsection{实验设置}
\paragraph{数据集与评价指标}
本文在\textbf{[数据集占位:例如 COCO/DOTA/VisDrone/自定义数据集名称]}上进行评估。若为目标检测任务,采用\textbf{[指标占位:mAP@0.5:0.95、AP50、AP75]};若为分割任务,采用\textbf{[指标占位:mIoU]};若为复原任务,采用\textbf{[指标占位:PSNR/SSIM]}。数据集划分与评测协议遵循\textbf{[协议占位:官方划分/自定义划分]}。

\paragraph{网络与替换策略}
我们选择\textbf{[主干/检测器占位:例如 RT-DETR/Deformable DETR/YOLO 系列]}作为基线。替换策略为:在\textbf{[位置占位:例如 backbone 的某些 depthwise conv、neck 的特定卷积层]}中,将原有深度卷积或 WTConv2d 替换为 AWTConv2d,同时保持其它层结构不变。为公平比较,除被替换模块外其余超参数保持一致。

\paragraph{训练细节}
训练采用\textbf{[优化器占位:SGD/AdamW]},初始学习率为\textbf{[lr 占位]},权重衰减为\textbf{[wd 占位]},batch size 为\textbf{[bs 占位]},训练轮数为\textbf{[epoch 占位]}。数据增强采用\textbf{[增强策略占位:多尺度、随机裁剪、mixup 等]}。实验运行在\textbf{[硬件占位:GPU 型号、显存]}上。

\subsection{对比方法}
对比方法包括:基线卷积(Depthwise Conv)、WTConv2d\cite{wtconv2024} 以及若干可插拔注意力/融合模块(例如坐标门控\cite{coordgate2024}、条带注意力\cite{fsa2024}、像素级融合\cite{cga2024} 等)。其中,所有对比均遵循相同的训练日程与推理设置。

\subsection{主结果}
\Cref{tab:main-results} 汇报了在\textbf{[数据集占位]}上的主结果。可以看到,在保持参数量与计算量变化可控的前提下,AWTConv2d 相比基线与 WTConv2d 均取得了稳定提升,说明所提出的坐标门控、条带频率门控以及像素级融合能够有效提升小波分支的自适应性与跨分支互补性。

\begin{table}[t]
\centering
\caption{在\textbf{[数据集占位]}上的主结果(数值为占位符,待补充)。}
\label{tab:main-results}
\begin{tabular}{lccc}
	oprule
方法 & 参数量(M) & FLOPs(G) & \textbf{[指标占位]} \\
\midrule
Baseline (DWConv) & [\#] & [\#] & [\#] \\
WTConv2d\cite{wtconv2024} & [\#] & [\#] & [\#] \\
AWTConv2d (Ours) & [\#] & [\#] & \best{[\#]} \\
\bottomrule
\end{tabular}
\end{table}

\subsection{消融实验}
为分析各组件的贡献,我们逐步加入 AWTConv2d 的关键设计,并保持其余设置不变。\Cref{tab:ablation} 给出了消融结果。总体趋势表现为:子带混合与子带注意力提供了稳健的基础增益;坐标门控进一步提升了不同空间位置的适配能力;条带频率门控对具有方向性纹理与边缘结构的样本更有帮助;像素级融合显著改善了空间分支与小波分支在局部区域的互补性,从而带来最终最优表现。

\begin{table}[t]
\centering
\caption{AWTConv2d 组件消融(数值为占位符,待补充)。}
\label{tab:ablation}
\begin{tabular}{lcccc}
	oprule
设置 & 子带混合 & CoordGate & StripGate & \textbf{[指标占位]} \\
\midrule
A0: WT骨架 + DW子带卷积 &  &  &  & [\#] \\
A1: + 子带混合 & \checkmark &  &  & [\#] \\
A2: + 坐标门控 & \checkmark & \checkmark &  & [\#] \\
A3: + 条带频率门控 & \checkmark & \checkmark & \checkmark & [\#] \\
A4: + 像素级融合(最终) & \checkmark & \checkmark & \checkmark & \best{[\#]} \\
\bottomrule
\end{tabular}
\end{table}

\subsection{定性分析与讨论(可选)}
为更直观地理解模块行为,我们建议可视化不同子带的注意力权重、坐标门控的空间分布以及像素融合门控 $\mathbf{P}$ 的热力图。若在目标检测任务中,可进一步对小目标、细长目标与复杂背景场景进行分组统计,观察条带频率门控对方向性纹理的贡献。对应的可视化结果可在\textbf{[可视化占位:图号/附录位置]}中补充。

\subsection{需要你提供的信息(用于把占位符替换为真实结果)}
为了将本节中的占位符替换为可发表的完整实验结果,我需要你确认:使用的数据集名称与划分、基线模型与配置文件路径、替换模块的具体位置、训练超参数(lr/epoch/bs)、以及最终的指标与计算量统计方式(例如用哪个脚本统计 FLOPs/Params)。


\section{Conclusion}
\label{sec:conclusion}

This paper proposes an adaptive wavelet convolution module, AWTConv2d, targeting plug-and-play integration of wavelet-based operators for vision tasks. While preserving the invertible wavelet analysis/synthesis scaffold, AWTConv2d improves adaptivity from both subband processing and cross-branch fusion. In the wavelet domain, we introduce per-channel learnable subband mixing and subband attention to enable content-adaptive reorganization among $LL/LH/HL/HH$. We further incorporate coordinate gating for position-conditioned modulation, and employ strip frequency gating to explicitly model directional low/high-frequency components. Finally, we adopt pixel-wise adaptive fusion to softly select between the spatial branch and the wavelet reconstruction branch at a fine granularity, mitigating the interference caused by naive summation.

Experimental results on \textbf{[Dataset Placeholder]} demonstrate that AWTConv2d provides consistent performance gains with a small additional computational cost, and ablation studies verify the effectiveness of each component. Future work may proceed in two directions. First, we will explore stronger forms of dynamic frequency filtering, such as low-rank or interpolatable frequency weights that generalize across resolutions. Second, we will extend the proposed adaptive mechanisms to more general multi-branch architectures and multimodal tasks, further evaluating robustness and interpretability in complex real-world scenarios.



\bibliographystyle{IEEEtran}
\bibliography{ref}
\end{document}


