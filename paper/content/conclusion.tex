\section{结论}
\label{sec:conclusion}

本文围绕小波卷积在视觉任务中的可插拔应用,提出了自适应小波卷积模块 AWTConv2d。该模块在保留小波分析/合成可逆骨架的前提下,从子带处理与跨分支融合两方面增强自适应性:在子带域内引入每通道的子带可学习混合与子带注意力,以实现对 $LL/LH/HL/HH$ 的内容自适应重组;进一步结合坐标门控实现位置条件化调制,并通过条带频率门控显式建模方向性低/高频成分;最后采用像素级自适应融合将空间分支与小波重建分支进行细粒度软选择,从而减少简单相加带来的互相干扰。

在\textbf{[数据集占位]}上的实验结果表明,AWTConv2d 能够在较小的额外计算开销下带来稳定的性能提升,并在消融实验中验证了各个组件的有效性。未来工作可从两方面展开:其一,探索更强的频域动态滤波形式,例如在保持分辨率泛化的前提下引入低秩或可插值的频域权重;其二,将本文的自适应机制推广到更通用的多分支架构与多模态任务,以进一步验证其在复杂场景下的稳健性与可解释性。