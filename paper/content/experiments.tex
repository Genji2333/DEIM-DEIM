\section{实验}
\label{sec:exp}

本节介绍实验设置、对比方法与结果分析。由于本文工作以模块替换为主,我们保持主干网络、训练策略与数据处理流程尽量一致,仅替换指定位置的卷积模块以验证 AWTConv2d 的通用增益。

\subsection{实验设置}
\paragraph{数据集与评价指标}
本文在\textbf{[数据集占位:例如 COCO/DOTA/VisDrone/自定义数据集名称]}上进行评估。若为目标检测任务,采用\textbf{[指标占位:mAP@0.5:0.95、AP50、AP75]};若为分割任务,采用\textbf{[指标占位:mIoU]};若为复原任务,采用\textbf{[指标占位:PSNR/SSIM]}。数据集划分与评测协议遵循\textbf{[协议占位:官方划分/自定义划分]}。

\paragraph{网络与替换策略}
我们选择\textbf{[主干/检测器占位:例如 RT-DETR/Deformable DETR/YOLO 系列]}作为基线。替换策略为:在\textbf{[位置占位:例如 backbone 的某些 depthwise conv、neck 的特定卷积层]}中,将原有深度卷积或 WTConv2d 替换为 AWTConv2d,同时保持其它层结构不变。为公平比较,除被替换模块外其余超参数保持一致。

\paragraph{训练细节}
训练采用\textbf{[优化器占位:SGD/AdamW]},初始学习率为\textbf{[lr 占位]},权重衰减为\textbf{[wd 占位]},batch size 为\textbf{[bs 占位]},训练轮数为\textbf{[epoch 占位]}。数据增强采用\textbf{[增强策略占位:多尺度、随机裁剪、mixup 等]}。实验运行在\textbf{[硬件占位:GPU 型号、显存]}上。

\subsection{对比方法}
对比方法包括:基线卷积(Depthwise Conv)、WTConv2d\cite{wtconv2024} 以及若干可插拔注意力/融合模块(例如坐标门控\cite{coordgate2024}、条带注意力\cite{fsa2024}、像素级融合\cite{cga2024} 等)。其中,所有对比均遵循相同的训练日程与推理设置。

\subsection{主结果}
\Cref{tab:main-results} 汇报了在\textbf{[数据集占位]}上的主结果。可以看到,在保持参数量与计算量变化可控的前提下,AWTConv2d 相比基线与 WTConv2d 均取得了稳定提升,说明所提出的坐标门控、条带频率门控以及像素级融合能够有效提升小波分支的自适应性与跨分支互补性。

\begin{table}[t]
\centering
\caption{在\textbf{[数据集占位]}上的主结果(数值为占位符,待补充)。}
\label{tab:main-results}
\begin{tabular}{lccc}
	oprule
方法 & 参数量(M) & FLOPs(G) & \textbf{[指标占位]} \\
\midrule
Baseline (DWConv) & [\#] & [\#] & [\#] \\
WTConv2d\cite{wtconv2024} & [\#] & [\#] & [\#] \\
AWTConv2d (Ours) & [\#] & [\#] & \best{[\#]} \\
\bottomrule
\end{tabular}
\end{table}

\subsection{消融实验}
为分析各组件的贡献,我们逐步加入 AWTConv2d 的关键设计,并保持其余设置不变。\Cref{tab:ablation} 给出了消融结果。总体趋势表现为:子带混合与子带注意力提供了稳健的基础增益;坐标门控进一步提升了不同空间位置的适配能力;条带频率门控对具有方向性纹理与边缘结构的样本更有帮助;像素级融合显著改善了空间分支与小波分支在局部区域的互补性,从而带来最终最优表现。

\begin{table}[t]
\centering
\caption{AWTConv2d 组件消融(数值为占位符,待补充)。}
\label{tab:ablation}
\begin{tabular}{lcccc}
	oprule
设置 & 子带混合 & CoordGate & StripGate & \textbf{[指标占位]} \\
\midrule
A0: WT骨架 + DW子带卷积 &  &  &  & [\#] \\
A1: + 子带混合 & \checkmark &  &  & [\#] \\
A2: + 坐标门控 & \checkmark & \checkmark &  & [\#] \\
A3: + 条带频率门控 & \checkmark & \checkmark & \checkmark & [\#] \\
A4: + 像素级融合(最终) & \checkmark & \checkmark & \checkmark & \best{[\#]} \\
\bottomrule
\end{tabular}
\end{table}

\subsection{定性分析与讨论(可选)}
为更直观地理解模块行为,我们建议可视化不同子带的注意力权重、坐标门控的空间分布以及像素融合门控 $\mathbf{P}$ 的热力图。若在目标检测任务中,可进一步对小目标、细长目标与复杂背景场景进行分组统计,观察条带频率门控对方向性纹理的贡献。对应的可视化结果可在\textbf{[可视化占位:图号/附录位置]}中补充。

\subsection{需要你提供的信息(用于把占位符替换为真实结果)}
为了将本节中的占位符替换为可发表的完整实验结果,我需要你确认:使用的数据集名称与划分、基线模型与配置文件路径、替换模块的具体位置、训练超参数(lr/epoch/bs)、以及最终的指标与计算量统计方式(例如用哪个脚本统计 FLOPs/Params)。

