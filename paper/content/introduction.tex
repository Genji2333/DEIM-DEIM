\section{引言}
\label{sec:intro}

卷积神经网络(CNN)在目标检测与密集预测任务中取得了显著进展,例如 DETR 系列方法通过全局匹配范式推动了端到端检测的发展\cite{carion2020detr,zhu2020deformable},而面向实时场景的检测器也不断在结构与训练策略上进行优化\cite{zhao2024rtdetr}。尽管如此,经典卷积算子本质上仍以局部感受野为核心,其对不同空间尺度、不同方向纹理以及不同频带信息的刻画往往依赖堆叠深层网络间接获得。对于含有丰富细节纹理、尺度跨度大或存在强背景干扰的场景,仅在空间域依赖固定卷积核的局部建模常会出现细节损失或结构误差积累。

频域方法为这一问题提供了另一种视角。相较于直接在空间域做卷积,频域表示可以更显式地区分低频结构与高频细节,从而在特征提取和特征融合阶段提供更强的可控性。近期研究中,动态频域滤波与频率感知融合等方向受到关注,例如基于 FFT 的动态滤波通过学习频域权重实现对不同样本的自适应混合\cite{dynamicfilter2023},频率感知特征融合通过显式低通/高通核生成增强跨尺度融合效果\cite{freqfusion2024}。然而,直接使用 FFT 在实际检测/分割框架中往往面临频域权重与分辨率绑定、边界处理复杂以及实现开销等问题。

小波变换兼具频域可解释性与空间局部性,能够以多尺度金字塔形式将特征分解为低频子带(结构)与高频子带(纹理),并可通过可逆重建将处理后的子带重新映射回空间域。基于小波的卷积模块(如 WTConv)在保持结构简单的同时提供了显式的频带通路\cite{wtconv2024}。但现有实现中,子带处理通常采用较静态的逐通道卷积与固定重标定:不同位置对频带的重要性难以动态调整,方向性低/高频响应缺乏显式建模,空间分支与小波分支的融合多为简单相加,容易在训练早期产生互相干扰。

本文提出一种自适应小波卷积 AWTConv2d,面向上述问题在小波子带处理与跨分支融合两方面进行增强。首先,我们在每个通道内部引入子带可学习混合,使 $\{LL,LH,HL,HH\}$ 的信息交换不再受限于固定子带语义;其次,借鉴坐标门控思想\cite{coordgate2024},使用由归一化坐标生成的空间门控对小波域特征进行逐点调制,以显式建模位置相关的频带贡献;同时引入条带频率门控\cite{fsa2024},对方向性低频/高频成分进行可学习重组,强化长条纹理与边缘结构建模;最后,在空间卷积分支与小波重建分支之间采用像素级自适应融合\cite{cga2024},避免简单相加带来的冲突,使网络能够在像素层面选择更可信的分支信息。

本文的主要贡献可概括为:我们提出一种可即插即用、接口与现有小波卷积兼容的自适应小波卷积模块;在不改变主干网络其余结构的前提下,通过坐标门控、条带频率门控与像素级融合提升了小波卷积对不同样本与不同空间区域的自适应性;并在\textbf{[数据集占位]}上通过主实验与消融实验验证了方法的有效性。