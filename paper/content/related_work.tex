\section{相关工作}
\label{sec:releated-work}

\subsection{小波变换与小波卷积}
小波变换通过一组分析滤波器将信号分解为不同尺度与不同方向的子带表示,兼具频域可解释性与空间局部性。与 FFT 不同,小波分解在空间上仍保持局部支持,因此更适合嵌入到卷积网络作为模块化算子。在视觉任务中,小波常用于多尺度表示与细节增强,近期也出现了将小波分析/合成嵌入卷积结构的工作,例如 WTConv 通过对特征进行多层小波分解,在各层子带上做轻量卷积后再逆变换重建,实现了较低改造成本的频带建模\cite{wtconv2024}。不过,现有小波卷积在子带处理与跨分支融合方面仍较为静态,限制了其对复杂场景的适配能力。

\subsection{频域建模与动态滤波}
频域建模常通过显式地对频率成分进行加权或滤波来增强结构/纹理表征。动态滤波思想通过内容自适应地生成频域权重,实现对不同输入的可变滤波响应\cite{dynamicfilter2023}。频率感知融合进一步关注跨尺度特征的低频/高频互补,通过生成低通/高通核对特征进行重采样与残差增强\cite{freqfusion2024}。这些方法证明了频域自适应的重要性,但也暴露出在检测与密集预测中直接使用 FFT 或复杂重采样算子的工程成本与分辨率绑定问题。

\subsection{空间门控与位置条件化调制}
位置条件化调制通常通过将外部条件(例如坐标、语义提示)映射为对特征的逐通道或逐点缩放,实现空间变化的响应函数。CoordGate 提出用坐标编码生成空间变化卷积的门控权重,从而在保持效率的同时提升空间可变性\cite{coordgate2024}。与之相关的条件化方法还包括 FiLM\cite{perez2018film}、AdaIN\cite{huang2017adain} 等,它们从更一般的角度说明了“对特征做可学习调制”能够有效增强模型的表达能力。

\subsection{注意力机制与特征融合}
注意力机制通过显式建模通道、空间或像素级权重以突出关键信息。针对双分支或多分支特征的融合,像素级注意力可以提供更细粒度的选择能力。CGA 融合模块通过通道注意力与空间注意力共同引导像素注意力,从而在两路特征之间实现逐像素的软融合\cite{cga2024}。该思路与本文的目标一致:当空间卷积分支与小波重建分支在不同区域的可靠性不同,像素级融合有助于减少简单相加导致的互相干扰。

综上,本文在小波卷积的分析/合成骨架上引入动态频域与空间调制机制,并采用像素级融合增强跨分支互补性,从而在保持模块可插拔的前提下提升自适应能力。