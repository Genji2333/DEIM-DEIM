\section{方法}
\label{sec:method}

本节给出小波卷积的基本形式,并详细介绍本文提出的自适应小波卷积模块 AWTConv2d。该模块在实现上保持与现有 WTConv2d 接口一致,便于在现有检测/分割网络中直接替换深度可分离卷积或深度卷积位置。

\subsection{预备:二维离散小波分析与合成}
给定输入特征图 $\mathbf{X}\in\mathbb{R}^{B\times C\times H\times W}$,二维离散小波变换可以视为一组固定滤波器的下采样卷积,其输出由一个低频子带与三个高频子带组成。本文将小波分析记为
\begin{equation}
\mathbf{U}=\mathcal{W}(\mathbf{X})\in\mathbb{R}^{B\times C\times 4\times \frac{H}{2}\times \frac{W}{2}},
\end{equation}
其中 $\mathbf{U}_{:,:,0,:,:}$ 对应 $LL$ 子带,$\mathbf{U}_{:,:,1:4,:,:}$ 对应 $LH/HL/HH$ 三个高频子带。小波合成(逆变换)记为
\begin{equation}
\hat{\mathbf{X}}=\mathcal{W}^{-1}(\mathbf{U})\in\mathbb{R}^{B\times C\times H\times W}.
\end{equation}
在工程实现上,$\mathcal{W}$ 与 $\mathcal{W}^{-1}$ 可分别由分组卷积与反卷积实现,滤波器由指定小波基(如 db1/db2)确定且参数固定。

\subsection{基线:WTConv2d 的子带处理与融合方式}
WTConv2d 的基本流程为多层小波分解、对子带特征做逐通道卷积、再逐层逆变换重建,最后将重建结果与空间深度卷积分支相加\cite{wtconv2024}。其核心优势在于显式频带通路与可逆重建,但其子带处理通常是相对静态的深度卷积与固定缩放,且融合仅为简单相加,缺乏输入自适应的选择机制。

\subsection{AWTConv2d:自适应小波子带 Token Mixer}
本文提出的 AWTConv2d 保留小波金字塔骨架,并将每层子带处理升级为包含“局部卷积 + 子带可学习混合 + 空间/方向调制 + 子带注意力”的组合算子。对第 $l$ 层小波输出 $\mathbf{U}^{(l)}$,我们先将其 reshape 为
\begin{equation}
\mathbf{T}^{(l)}\in\mathbb{R}^{B\times 4C\times H_l\times W_l},
\end{equation}
其中 $H_l=H/2^l,\,W_l=W/2^l$。

\paragraph{(1)子带内局部建模}
首先对 $\mathbf{T}^{(l)}$ 施加逐通道卷积以捕获每个子带内部的局部模式,记为 $\phi_{dw}(\cdot)$。

\paragraph{(2)每通道的子带可学习混合}
为了允许同一通道的 $\{LL,LH,HL,HH\}$ 之间进行信息交换,我们引入按通道分组的 $1\times1$ 变换 $\phi_{mix}(\cdot)$,其 groups 设为 $C$,从而在每个通道内学习一个 $4\rightarrow 4$ 的线性混合。与对 $4C$ 统一混合不同,该设计避免跨通道的无约束耦合,保持稳定性并提高可解释性。

\paragraph{(3)坐标门控:位置条件化的频带调制}
不同空间位置对低频/高频的依赖通常不同,例如目标边缘区域更依赖高频细节,背景大面积平坦区域更依赖低频结构。借鉴 CoordGate 的思想\cite{coordgate2024},我们以归一化坐标网格 $(x,y)\in[-1,1]^2$ 作为条件输入,经由轻量 $1\times1$ 卷积网络生成逐像素门控 $\mathbf{G}_{coord}^{(l)}\in(0,1)^{B\times 4C\times H_l\times W_l}$,并进行逐点调制:
\begin{equation}
	ilde{\mathbf{T}}^{(l)} = \mathbf{T}^{(l)} \odot \mathbf{G}_{coord}^{(l)}.
\end{equation}
该实现不依赖固定输入分辨率,因此可适配不同 stage 与不同尺寸输入。

\paragraph{(4)条带频率门控:方向性低/高频重组}
为了显式建模方向性纹理与长条结构,我们引入条带低/高频分解\cite{fsa2024}。具体地,沿水平方向与垂直方向分别进行条带平均池化得到低频分量,并以残差形式得到高频分量;随后通过可学习系数对低/高频混合并以残差方式回注。记该算子为 $\psi_{strip}(\cdot)$,则有
\begin{equation}
\bar{\mathbf{T}}^{(l)} = \psi_{strip}(\tilde{\mathbf{T}}^{(l)}).
\end{equation}
与仅依赖卷积核隐式学习方向响应不同,该机制以结构化方式提供了对“低频平滑”与“高频边缘”的可控重加权。

\paragraph{(5)子带注意力:输入自适应的子带重标定}
在子带混合与空间/方向调制之后,我们进一步对 $\bar{\mathbf{T}}^{(l)}$ 施加子带注意力。该注意力通过全局池化获取内容统计,并采用 groups=$C$ 的 $1\times1$ 变换在每个通道内部生成 $4$ 个子带权重,实现对 $LL/LH/HL/HH$ 的动态选择。

综上,第 $l$ 层子带处理可概括为
\begin{equation}
\mathbf{T}^{(l)} \leftarrow \alpha\big(\psi_{strip}(\phi_{coord}(\phi_{mix}(\phi_{dw}(\mathbf{T}^{(l)}))))\big),
\end{equation}
其中 $\alpha(\cdot)$ 表示子带注意力与缩放。

\subsection{小波重建与跨分支像素级融合}
经过 $L$ 层子带处理后,我们自顶向下执行逐层逆变换重建,得到小波分支输出 $\mathbf{X}_{wave}$。与此同时,空间分支采用深度卷积得到 $\mathbf{X}_{base}$。

以往方法常直接相加 $\mathbf{X}_{base}+\mathbf{X}_{wave}$,但在不同区域两分支的可靠性并不一致。为此,我们引入像素级融合模块,借鉴内容引导注意力融合思想\cite{cga2024},综合通道注意力与空间注意力生成像素门控 $\mathbf{P}\in(0,1)^{B\times C\times H\times W}$,并进行逐像素软融合:
\begin{equation}
\mathbf{X}_{fuse} = \mathbf{X}_{init} + \mathbf{P}\odot\mathbf{X}_{base} + (1-\mathbf{P})\odot\mathbf{X}_{wave},\quad \mathbf{X}_{init}=\mathbf{X}_{base}+\mathbf{X}_{wave}.
\end{equation}
此外,我们保留一个保守的逐通道门控 $\sigma(\mathbf{g})$ 先对 $\mathbf{X}_{wave}$ 做幅度调制,使训练初期不至于因小波分支过强而不稳定。最终输出再通过 $1\times1$ 卷积调整通道数以匹配下游网络。

\subsection{复杂度与可插拔性讨论}
AWTConv2d 在 WTConv2d 的基础上新增的主要开销来自于子带混合的 $1\times1$ 分组卷积、坐标门控的轻量网络、条带池化与像素级融合中的 depthwise $7\times7$。这些算子均可在标准深度学习框架中高效实现,不依赖额外的 CUDA 自定义算子;同时模块保持与 WTConv2d 相同的输入输出接口,因而可直接用于现有工程代码中进行替换与消融。