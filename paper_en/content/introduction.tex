\section{Introduction}
\label{sec:intro}

Convolutional neural networks (CNNs) have achieved remarkable progress in object detection and dense prediction. For instance, the DETR family advances end-to-end detection through global matching paradigms~\cite{carion2020detr,zhu2020deformable}, and real-time detectors continue to be improved in both architecture and training strategies~\cite{zhao2024rtdetr}. Nevertheless, classical convolution operators are still centered around local receptive fields; their capability to model multi-scale patterns, directional textures, and frequency-band information is often obtained indirectly via deep stacking. In scenes with rich textures, large scale variations, or heavy background clutter, relying only on spatial-domain local modeling with fixed kernels may lead to detail loss or accumulated structural errors.

Frequency-domain approaches offer an alternative perspective. Compared with spatial-domain convolutions, frequency representations can more explicitly separate low-frequency structures from high-frequency details, providing better controllability for feature extraction and feature fusion. Recent studies have explored dynamic frequency filtering and frequency-aware fusion. For example, FFT-based dynamic filtering learns frequency-domain weights to adaptively mix responses across samples~\cite{dynamicfilter2023}, while frequency-aware feature fusion explicitly generates low-pass and high-pass kernels to enhance cross-scale fusion~\cite{freqfusion2024}. However, directly adopting FFT in practical detection/segmentation pipelines often faces challenges such as resolution-dependent frequency weights, complicated boundary handling, and non-trivial implementation overhead.

Wavelet transforms combine frequency interpretability with spatial locality. They decompose features into multi-scale pyramids of low-frequency subbands (structures) and high-frequency subbands (textures), and can map processed subbands back to the spatial domain via invertible reconstruction. Wavelet-based convolution modules (e.g., WTConv) provide explicit band pathways while remaining structurally simple~\cite{wtconv2024}. Yet, in many existing implementations, subband processing is relatively static, typically consisting of depthwise operations and fixed re-scaling. Consequently, the importance of different bands is hard to adjust dynamically across locations; directional low/high-frequency responses are not explicitly modeled; and the fusion between the spatial branch and the wavelet branch is commonly a simple summation, which may cause branch interference in early training.

To address these issues, we propose an adaptive wavelet convolution module, AWTConv2d, which enhances both wavelet-subband processing and cross-branch fusion. First, we introduce per-channel learnable subband mixing so that information exchange among $\{LL, LH, HL, HH\}$ is no longer restricted by fixed subband semantics. Second, inspired by coordinate gating~\cite{coordgate2024}, we generate spatial gates from normalized coordinates to modulate wavelet-domain features, explicitly modeling location-dependent band contributions. Meanwhile, we introduce strip frequency gating~\cite{fsa2024} to reorganize directional low/high-frequency components and strengthen the modeling of elongated textures and edge structures. Finally, we employ pixel-wise adaptive fusion~\cite{cga2024} between the spatial convolution branch and the wavelet reconstruction branch, avoiding conflicts caused by naive summation and enabling per-pixel selection of more reliable information.

Our contributions are summarized as follows. We present a plug-and-play adaptive wavelet convolution module that remains interface-compatible with existing wavelet convolutions. Without altering the rest of the backbone, the proposed coordinate gating, strip frequency gating, and pixel-wise fusion significantly improve adaptivity across samples and spatial regions. Extensive experiments on \textbf{[Dataset Placeholder]} with both main results and ablations validate the effectiveness of the proposed design.
